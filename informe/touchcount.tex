\section{Touch Count}

\subsection{Motivación}

However, there is a cache killer called the full-table scan where each and every table block is
placed into the buffer. If the buffer cache is 500 blocks and the table contains 600 blocks, all the
popular blocks would be replaced with this full-table scanned table’s blocks. This is extremely
disruptive to consistent database performance because it forces excessive computer system usage
and basically destroys a well-developed cache.


While this might seem like all our buffer cache problems are solved, think again. How about a
large index range scan? Picture hundreds of index leaf blocks flowing into the buffer cache. The
modified LRU algorithm only addresses full-table scan issues, not index block issues.


\subsection{Descripción}
El algoritmo que utiliza Oracle para manejar las páginas del Buffer Pool es conocido
como “Touch Count” y es una variante del popular LRU.

\subsubsection{Hot N Cold}

Que significa\\
Como se mantiene balanceado \\
Como se crea\\

\subsubsection{Incremento Touch Count}

Lo de los 3 segundos\\
pin y unpin\\

\subsubsection{Hot N Cold Movement}

Si es mayor a 2\\
Mantener balanceo\\

\subsection{Implementación}
\subsubsection{Touch Count Buffer Frame}

\begin{lstlisting}
package ubadb.core.components.bufferManager.bufferPool.replacementStrategies.touchcount;

import java.util.Date;

import ubadb.core.common.Page;
import ubadb.core.components.bufferManager.bufferPool.BufferFrame;
import ubadb.core.exceptions.BufferFrameException;

public class TouchBufferFrame extends BufferFrame implements Comparable<TouchBufferFrame>{
        
		public Integer count;
		public Date lastTouch;
	
        public TouchBufferFrame(Page page) {
                super(page);
                count = 0;
                lastTouch = new Date();
        }
        
        public void pin(){
        	super.pin();
        	increaseCount();
        }
        
        public void unpin() throws BufferFrameException{
        	super.unpin();
        	increaseCount();
        }
        
        public void increaseCount(){
        	Date now = new Date();
        	
        	@SuppressWarnings("unused")
			long difference = (long) ((now.getTime() - lastTouch.getTime())/1000);
        	
        	if((now.getTime() - lastTouch.getTime())/1000 >= 3){
        		count++;
        		lastTouch = new Date();
        	}
        }

		@Override
		public int compareTo(TouchBufferFrame arg0) {
			return count.compareTo(((TouchBufferFrame)arg0).count);
		}

}
\end{lstlisting}

\subsubsection{Touch Count Buffer Frame Strategy}

\begin{lstlisting}
package ubadb.core.components.bufferManager.bufferPool.replacementStrategies.touchcount;

import java.util.Collection;
import java.util.LinkedList;

import ubadb.core.common.Page;
import ubadb.core.components.bufferManager.bufferPool.BufferFrame;
import ubadb.core.components.bufferManager.bufferPool.
	replacementStrategies.PageReplacementStrategy;
import ubadb.core.exceptions.PageReplacementStrategyException;

public class TouchCountReplacementStrategy implements PageReplacementStrategy {

		private LinkedList<TouchBufferFrame> cold;
		private LinkedList<TouchBufferFrame> hot;
	
        public BufferFrame findVictim(Collection<BufferFrame> bufferFrames) throws PageReplacementStrategyException {
       		
        		hotNColdMovement();
        		
        		TouchBufferFrame victim = firstColdFrame();
                return victim;
        }
        
        private TouchBufferFrame firstColdFrame() throws PageReplacementStrategyException{
        	for (TouchBufferFrame bufferFrame : cold) {
        		if (bufferFrame.canBeReplaced()){
        			cold.remove(bufferFrame);
        			
        			if(Math.abs(cold.size() - hot.size())>0){ 
        				cold.addLast(hot.removeLast());
        			}
        			
        			return bufferFrame;
        		}
        	}
        	throw new PageReplacementStrategyException("No hay Cold Buffer reemplazable");
        }
        
        private void hotNColdMovement(){
        	for(TouchBufferFrame bufferFrame : cold){
        		if(bufferFrame.count > 2 ){ 
        			bufferFrame.count = 0;
        			hot.addFirst(bufferFrame);
        			cold.remove(bufferFrame);
        			cold.addLast(hot.removeLast()); 
        		}
        	}
        	
        }
        

        public BufferFrame createNewFrame(Page page) {
        	TouchBufferFrame bufferFrame = new TouchBufferFrame(page);
        	
        	cold.addFirst(bufferFrame);
        	
        	if (cold.size() >= 2){ 
        		TouchBufferFrame coldToHodFrame = cold.pop();
        		hot.addLast(coldToHodFrame);
        	}
        	
        	return bufferFrame;     
        	
        }
        
        public String toString() {
                return "Touch Count Replacement Strategy";
        }
        
}
\end{lstlisting}
