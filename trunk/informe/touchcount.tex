\section{Touch Count}

\subsection{Motivación}

Tanto la estrategia MRU como la LRU aplican un mal uso de la caché en los casos en que se realizan full scans de las tablas
debido a que se cargan todos los bloques de éstas en el buffer. Por ejemplo, si el buffer es de 500 bloques y la tabla contiene 600,
todos los bloques populares serán removidos y reemplazados con los bloques de esta tabla. Esto es sumamente contraproducente para la consistencia de la performance de la base de datos ya que fuerza un excesivo uso del sistema y destruye una cache bien conformada.
\\
Para solucionar este problema existe la famosa estrategia MLRU (modified least recently used). Esta lo que hace es ubicar los bloques
leídos por un full-scan al final de la cola limitando, adicionalmente, la cantidad de bloques para ello.
\\
Sin embargo esta estrategia no funciona para index-scan y si debemos realizar dicho sobre un amplio rango volvemos al mismo problema. Para
esto se pensó la estrategia Touch-Count, que será desarrollada a lo largo de esta sección.

\subsection{Descripción}
El algoritmo que utiliza Oracle para manejar las páginas del Buffer Pool es conocido
como “Touch Count” y es una variante del popular LRU.

\subsubsection{Hot N Cold}

Este algoritmo contiene una lista de buffers dividida en dos secciones, una denominada fría (cold) y otra caliente (hot), donde ambas tienen la misma cantidad de bloques (en caso de ser impar ésta, la zona fría tiene ese bloque de más)\\
La idea es que esta lista se mantenga balanceada en base a cuántos accesos tuvo, manteniendo en la sección caliente los más "populares". \\
Cada vez que se agrega un nuevo bloque a la lista, este se introduce en el medio, siempre en la parte fría. En caso de quedar esta con una diferencia mayor a 1 con respecto a la caliente se debe balancear nuevamente la lista.\\

\subsubsection{Increase Touch Count }

Los accesos a los bloques se contabilizan a través de un valor denominado "Touch Count" que en la práctica no es estrictamente la 
cantidad de acccesos. Por ejemplo este valor sólo puede ser actualizado una vez cada 3 segundos para no sobrecargar los accesos a memoria y también por practicidad el valor sólo es incrementando en el unpin (cuando se "suelta" el bloque), porque sino estarían duplicados innecesariamente. Además estos valores son "reseteados" cuando un bloque se mueve de zona.\\


\subsubsection{Hot N Cold Movement}

Un bloque pasa de la zona fría a la caliente y su Touch Count se reinicia a 0 cuando éste es mayor a 2 (o durante un balanceo como se mencionó anteriormente). Además el último bloque hot también es movido y su Touch Count pasa a 1.\\

\subsection{Implementación}
\subsubsection{Touch Count Buffer Frame}

\begin{lstlisting}
package ubadb.core.components.bufferManager.bufferPool.replacementStrategies.touchcount;

import java.util.Date;

import ubadb.core.common.Page;
import ubadb.core.components.bufferManager.bufferPool.BufferFrame;
import ubadb.core.exceptions.BufferFrameException;

public class TouchBufferFrame extends BufferFrame implements Comparable<TouchBufferFrame>{
        
		public Integer count;
		public Date lastTouch;
	
        public TouchBufferFrame(Page page) {
                super(page);
                count = 0;
                lastTouch = new Date();
        }
        
        public void pin(){
        	super.pin();
        	increaseCount();
        }
        
        public void unpin() throws BufferFrameException{
        	super.unpin();
        	increaseCount();
        }
        
        public void increaseCount(){
        	Date now = new Date();
        	
        	@SuppressWarnings("unused")
			long difference = (long) ((now.getTime() - lastTouch.getTime())/1000);
        	
        	if((now.getTime() - lastTouch.getTime())/1000 >= 3){
        		count++;
        		lastTouch = new Date();
        	}
        }

		@Override
		public int compareTo(TouchBufferFrame arg0) {
			return count.compareTo(((TouchBufferFrame)arg0).count);
		}

}
\end{lstlisting}

\subsubsection{Touch Count Buffer Frame Strategy}

\begin{lstlisting}
package ubadb.core.components.bufferManager.bufferPool.replacementStrategies.touchcount;

import java.util.Collection;
import java.util.LinkedList;

import ubadb.core.common.Page;
import ubadb.core.components.bufferManager.bufferPool.BufferFrame;
import ubadb.core.components.bufferManager.bufferPool.replacementStrategies.PageReplacementStrategy;
import ubadb.core.exceptions.PageReplacementStrategyException;

public class TouchCountReplacementStrategy implements PageReplacementStrategy {

		private LinkedList<TouchBufferFrame> cold;
		private LinkedList<TouchBufferFrame> hot;
		
		public TouchCountReplacementStrategy(){
			cold = new LinkedList<TouchBufferFrame>();
			hot = new LinkedList<TouchBufferFrame>();
		}
		
        public BufferFrame findVictim(Collection<BufferFrame> bufferFrames) throws PageReplacementStrategyException {
        		hotNColdMovement(); // Mover según parámetro del paper.
        		
        		TouchBufferFrame victim = firstColdFrame();
                return victim;
        }
        
        //Devuelvo el primer COLD Frame reemplazable
        private TouchBufferFrame firstColdFrame() throws PageReplacementStrategyException{
        	for (TouchBufferFrame bufferFrame : cold) {
        		if (bufferFrame.canBeReplaced()){
        			cold.remove(bufferFrame);
        			
        			if(Math.abs(cold.size() - hot.size())>0){ //Si HOT es más grande
        				cold.addLast(hot.removeLast()); //Mantener balanceo
        			}//Si no, es que son iguales. No puede haber más de 1 de diferencia
        			
        			return bufferFrame;
        		}
        	}
        	throw new PageReplacementStrategyException("No hay Cold Buffer reemplazable");
        }
        
        //Una vez que tengamos cold y hoy, pasar de cold a hot según parámetro del paper
        private void hotNColdMovement(){
			LinkedList<TouchBufferFrame> coldToRemove = new LinkedList<TouchBufferFrame>();
			LinkedList<TouchBufferFrame> hotToRemove = new LinkedList<TouchBufferFrame>();
        	for(TouchBufferFrame bufferFrame : cold){
        		if(bufferFrame.count >= 2 ){ // Dato del paper
        			bufferFrame.count = 0; //Se tiene que resetear
        			
        			coldToRemove.addLast(bufferFrame);
        			
        			TouchBufferFrame toColdToMantainBalanceFrame = hot.removeLast();
					toColdToMantainBalanceFrame.count = 1; //Si no en la proxima iteracion lo va a mandar a Hot de vuelta
					hotToRemove.addLast(toColdToMantainBalanceFrame); //Mantener balanceo
        		}  		
        	}
        	
    		//Elimino los que tienen count>2 de cold
    		cold.removeAll(coldToRemove);
    		//Concateno de forma tal que quede al principio coldToRemove y después hot
    		coldToRemove.addAll(hot);
    		//Hot es esta nueva lista
    		hot = coldToRemove;
    		
    		//MantenerBalanceo agregando los que saqué de Hot al final de cold 
    		cold.addAll(hotToRemove);
        }

        public BufferFrame createNewFrame(Page page) {
        	TouchBufferFrame bufferFrame = new TouchBufferFrame(page);
        	
        	cold.addFirst(bufferFrame);
        	
        	if (Math.abs(cold.size() - hot.size())>1){ //Mantener balanceo, máxima diferencia 1
        		TouchBufferFrame coldToHodFrame = cold.removeLast();
        		hot.addLast(coldToHodFrame);
        	}
        	
        	return bufferFrame;     
        }
        
        public String toString() {
            return "Touch Count Replacement Strategy";
        }
}
\end{lstlisting}
