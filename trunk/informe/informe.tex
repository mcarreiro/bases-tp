\documentclass[a4paper,titlepage,10pt]{article}

\usepackage[margin=0.6in]{geometry} % margenes
\usepackage[spanish]{babel} % Le indicamos a LaTeX que vamos a escribir en español.
\usepackage[utf8]{inputenc} % Quiero acentos
\usepackage{caratula}
\usepackage{listings} % Java Typeset
\usepackage{color} % Para definir color

\definecolor{pblue}{rgb}{0.13,0.13,1}
\definecolor{pgreen}{rgb}{0,0.5,0}
\definecolor{pred}{rgb}{0.9,0,0}
\definecolor{pgrey}{rgb}{0.46,0.45,0.48}

\lstset{language=Java,
  showspaces=false,
  showtabs=false,
  breaklines=true,
  showstringspaces=false,
  breakatwhitespace=true,
  commentstyle=\color{pgreen},
  keywordstyle=\color{pblue},
  stringstyle=\color{pred},
  basicstyle=\ttfamily,
}

\titulo{Trabajo Práctico 2}
\fecha{15 / 11 / 2013}
\materia{Bases De Datos}
\grupo{Grupo 4}
\integrante{Carreiro, Martin}{45/10}{martin301290@gmail.com}
\integrante{Kujawski, Kevin}{459/10}{kevinkuja@gmail.com}
\integrante{Ortiz De Zarate, Juan Manuel}{403/10}{jmanuoz@gmail.com}
\integrante{Teren, Leonardo}{332/09}{lteren@gmail.com}

\begin{document} % Todo lo que escribamos a partir de aca va a aparecer en el documento.

\maketitle

\section{Codigo}

\section{Introducción}

El Buffer Manager es uno de los componentes más importantes dentro de un
motor de BD. Su principal función es administrar un espacio de memoria de la BD,
utilizado como una especie de memoria caché. El objetivo es que las diferentes
aplicaciones que usan la BD y requieren páginas de disco, puedan recuperar la página de
este espacio de memoria y accedan lo menos posible al disco.
El espacio de memoria administrado por el Buffer Manager puede ser organizado de
diferentes formas y la estrategia para decidir cuál página reemplazar cuando ya no queda
más espacio también puede variar.\\


Este trabajo se encarga de implementar los algoritmos Touch Count, LRU y MRU, y compararemos su comportamiento 
a partir de posibles situaciones.


\section{LRU}

\subsection{Descripción}

El algoritmo Least Recently Used (LRU) descarta primero los elementos menos usados recientemente. El algoritmo lleva el seguimiento de lo que se va usando, lo que resulta caro si se quiere hacer con precisión. La implementación de esta técnica requiere llevar la cuenta de la edad de cada elemento de caché y buscar el menos usado en base a ella. En una implementación como esa, cada vez que se usa un elemento, la edad de todos las demás elementos cambia.

\subsection{Implementación}

\subsubsection{LRU Buffer Frame}

\begin{lstlisting}
package ubadb.core.components.bufferManager.bufferPool.replacementStrategies.lru;

import java.util.Date;

import ubadb.core.common.Page;
import ubadb.core.components.bufferManager.bufferPool.BufferFrame;
import ubadb.core.exceptions.BufferFrameException;

public class LRUBufferFrame extends BufferFrame 
{                
		private Date referencedDate;
	
        public LRUBufferFrame(Page page) {
                super(page);     
                referencedDate = new Date();
        }
        
        public Date getReferencedDate()
    	{
    		return referencedDate;
    	}
        
        public void pin(){
        	super.pin();
        	referencedDate = new Date();
        }
        
        public void unpin() throws BufferFrameException{
        	super.unpin();
        	referencedDate = new Date();
        }


}
\end{lstlisting}

\subsubsection{LRU Buffer Frame Strategy}

\begin{lstlisting}
package ubadb.core.components.bufferManager.bufferPool.replacementStrategies.lru;

import java.util.Collection;
import java.util.Date;

import ubadb.core.common.Page;
import ubadb.core.components.bufferManager.bufferPool.BufferFrame;
import ubadb.core.components.bufferManager.bufferPool.
	replacementStrategies.PageReplacementStrategy;
import ubadb.core.exceptions.PageReplacementStrategyException;

public class LRUReplacementStrategy implements PageReplacementStrategy
{       
    public BufferFrame findVictim(Collection<BufferFrame> bufferFrames) throws PageReplacementStrategyException
    {               
    	LRUBufferFrame victim = null;
		Date oldestReplaceablePageDate = null;
		
		for(BufferFrame bufferFrame : bufferFrames)
		{
			LRUBufferFrame lruBufferFrame = (LRUBufferFrame) bufferFrame; //safe cast as we know all frames are of this type
			if(lruBufferFrame.canBeReplaced() && (oldestReplaceablePageDate==null || 
				lruBufferFrame.getReferencedDate().
					before(oldestReplaceablePageDate)))
			{
				victim = lruBufferFrame;
				oldestReplaceablePageDate = lruBufferFrame.getReferencedDate();
			}
		}
		
		if(victim == null)
			throw new PageReplacementStrategyException("No page can be removed from pool");
		else
			return victim;
    }
    
    public BufferFrame createNewFrame(Page page) 
    {
    	return new LRUBufferFrame(page);                     
    }
    
    public String toString() {
    	return "LRU Replacement Strategy";
    }
}
\end{lstlisting}


\newpage

\section{MRU}

\subsection{Descripción}

MRU descarta primero -al contrario de LRU- los elementos más usados recientemente.
Los algoritmos MRU son los más útiles en situaciones en las que cuanto más entiguo es un elemento, más probable es que se acceda a él.

\subsection{Implementación}
\subsubsection{MRU Buffer Frame}

\begin{lstlisting}
package ubadb.core.components.bufferManager.bufferPool.replacementStrategies.mru;
import ubadb.core.common.Page;
import ubadb.core.components.bufferManager.bufferPool.BufferFrame;
import ubadb.core.exceptions.BufferFrameException;

import java.util.Date;

public class MRUBufferFrame extends BufferFrame 
{
		private Date referencedDate;
	
        public MRUBufferFrame(Page page) {
                super(page); 
                referencedDate = new Date();
        }
        
        public Date getReferencedDate()
    	{
    		return referencedDate;
    	}
        
        public void pin(){
        	super.pin();
        	referencedDate = new Date();
        }
        
        public void unpin() throws BufferFrameException{
        	super.unpin();
        	referencedDate = new Date();
        }

}
\end{lstlisting}

\subsubsection{MRU Buffer Frame Strategy}

\begin{lstlisting}
package ubadb.core.components.bufferManager.bufferPool.replacementStrategies.mru;

import java.util.Collection;
import java.util.Date;

import ubadb.core.common.Page;
import ubadb.core.components.bufferManager.bufferPool.BufferFrame;
import ubadb.core.components.bufferManager.bufferPool.
	replacementStrategies.PageReplacementStrategy;
import ubadb.core.exceptions.PageReplacementStrategyException;

public class MRUReplacementStrategy implements PageReplacementStrategy
{       
    public BufferFrame findVictim(Collection<BufferFrame> bufferFrames) throws PageReplacementStrategyException
    {               
    	MRUBufferFrame victim = null;
		Date newestReplaceablePageDate = null;
		
		for(BufferFrame bufferFrame : bufferFrames)
		{
			MRUBufferFrame mruBufferFrame = (MRUBufferFrame) bufferFrame; //safe cast as we know all frames are of this type
			if(mruBufferFrame.canBeReplaced() && (newestReplaceablePageDate==null 
				|| mruBufferFrame.getReferencedDate().
					after(newestReplaceablePageDate)))
			{
				victim = mruBufferFrame;
				newestReplaceablePageDate = mruBufferFrame.getReferencedDate();
			}
		}
		
		if(victim == null)
			throw new PageReplacementStrategyException("No page can be removed from pool");
		else
			return victim;
    }
    
    public BufferFrame createNewFrame(Page page) 
    {
    	return new MRUBufferFrame(page);               
    }
    
    public String toString() {
            return "MRU Replacement Strategy";
    }
}
\end{lstlisting}


\newpage

\section{Touch Count}

\subsection{Descripción}
El algoritmo que utiliza Oracle para manejar las páginas del Buffer Pool es conocido
como “Touch Count” y es una variante del popular LRU.

\subsubsection{Hot N Cold}

\subsubsection{Incremento Touch Count}

\subsubsection{Hot N Cold Movement}

\subsection{Implementación}
\subsubsection{Touch Count Buffer Frame}

\begin{lstlisting}
package ubadb.core.components.bufferManager.bufferPool.replacementStrategies.touchcount;

import java.util.Date;

import ubadb.core.common.Page;
import ubadb.core.components.bufferManager.bufferPool.BufferFrame;
import ubadb.core.exceptions.BufferFrameException;

public class TouchBufferFrame extends BufferFrame implements Comparable<TouchBufferFrame>{
        
		public Integer count;
		public Date lastTouch;
	
        public TouchBufferFrame(Page page) {
                super(page);
                count = 0;
                lastTouch = new Date();
        }
        
        public void pin(){
        	super.pin();
        	increaseCount();
        }
        
        public void unpin() throws BufferFrameException{
        	super.unpin();
        	increaseCount();
        }
        
        public void increaseCount(){
        	Date now = new Date();
        	
        	@SuppressWarnings("unused")
			long difference = (long) ((now.getTime() - lastTouch.getTime())/1000);
        	
        	if((now.getTime() - lastTouch.getTime())/1000 >= 3){
        		count++;
        		lastTouch = new Date();
        	}
        }

		@Override
		public int compareTo(TouchBufferFrame arg0) {
			return count.compareTo(((TouchBufferFrame)arg0).count);
		}

}
\end{lstlisting}

\subsubsection{Touch Count Buffer Frame Strategy}

\begin{lstlisting}
package ubadb.core.components.bufferManager.bufferPool.replacementStrategies.touchcount;

import java.util.Collection;
import java.util.LinkedList;

import ubadb.core.common.Page;
import ubadb.core.components.bufferManager.bufferPool.BufferFrame;
import ubadb.core.components.bufferManager.bufferPool.
	replacementStrategies.PageReplacementStrategy;
import ubadb.core.exceptions.PageReplacementStrategyException;

public class TouchCountReplacementStrategy implements PageReplacementStrategy {

		private LinkedList<TouchBufferFrame> cold;
		private LinkedList<TouchBufferFrame> hot;
	
        public BufferFrame findVictim(Collection<BufferFrame> bufferFrames) throws PageReplacementStrategyException {
       		
        		hotNColdMovement();
        		
        		TouchBufferFrame victim = firstColdFrame();
                return victim;
        }
        
        private TouchBufferFrame firstColdFrame() throws PageReplacementStrategyException{
        	for (TouchBufferFrame bufferFrame : cold) {
        		if (bufferFrame.canBeReplaced()){
        			cold.remove(bufferFrame);
        			
        			if(Math.abs(cold.size() - hot.size())>0){ 
        				cold.addLast(hot.removeLast());
        			}
        			
        			return bufferFrame;
        		}
        	}
        	throw new PageReplacementStrategyException("No hay Cold Buffer reemplazable");
        }
        
        private void hotNColdMovement(){
        	for(TouchBufferFrame bufferFrame : cold){
        		if(bufferFrame.count > 2 ){ 
        			bufferFrame.count = 0;
        			hot.addFirst(bufferFrame);
        			cold.remove(bufferFrame);
        			cold.addLast(hot.removeLast()); 
        		}
        	}
        	
        }
        

        public BufferFrame createNewFrame(Page page) {
        	TouchBufferFrame bufferFrame = new TouchBufferFrame(page);
        	
        	cold.addFirst(bufferFrame);
        	
        	if (cold.size() >= 2){ 
        		TouchBufferFrame coldToHodFrame = cold.pop();
        		hot.addLast(coldToHodFrame);
        	}
        	
        	return bufferFrame;     
        	
        }
        
        public String toString() {
                return "Touch Count Replacement Strategy";
        }
        
}
\end{lstlisting}


\newpage

\section{Comparaciones}
Usamos los problemas mencionados en TouchCount


\section{Conclusión}


\end{document} %Terminé!
